% Place your vita below.
Tobias Frueh was born in Ellwangen/Jagst, Germany in 1990 and was raised in Ammelbruch, Germany. Tobias received his Bachelor of Science degree in Nanoscience and Technology in May 2013 and his Master of Science degree in Materials Science and Engineering in January 2015, both from the University of Erlangen-Nuremberg, Germany. During his Masters education, he studied for one year at The Pennsylvania State University, USA to conduct the research for his Masters Thesis and was co-advised by Professor Andreas Roosen and Professor Gary Messing. After completing his Mater of Science degree Tobias continued his graduate studies at The Pennsylvania State University and was advised by Professor Gary L. Messing. After graduation, Tobias joined Almatis GmbH in Ludwigshafen, Germany as Application and Market Development Engineer.
\\
\newline
\noindent List of Publications:
\begin{itemize}
	\item Tobias Frueh, Elizabeth R. Kupp, Charles Compson, Joe Atria, Nils Rosenberger and Gary L. Messing, "Modal interpretation of Sintering Kinetics Based on Dilatometry," chi/Ber. DKG, 92, E1-4 (2015). 
	\item Tobias Frueh, Elizabeth R. Kupp, Charles Compson, Joe Atria and Gary L. Messing, "The Effect of Na$_{2}$O and SiO$_{2}$ on Liquid Phase Sintering of Bayer Al$_{2}$O$_{3}$," J. Am. Ceram. Soc., 99, 2267-2272 (2016).
	\item Gary L. Messing, Stephen Poterala, Yunfei Chang, Tobias Frueh, Elizabeth R. Kupp, Beecher H. Watson III, Rebecca L. Walton, Michael J. Brova, Anna-Katharina Hofer, Raul Bermejo, Richard J. Meyer, "Texture-engineered ceramics - Property enhancements through crystallographic tailoring" Journal of Materials Research, https://doi.org/10.1557/jmr.2017.207 (2017).
	\item Tobias Frueh, Cassie Marker, Elizabeth R. Kupp, Charles Compson, Joe Atria, Jennifer L. Gray, Zi-Kui Liu, and Gary L. Messing, "Powder Chemistry Effects on the Sintering of MgO-doped Specialty Al2O3," accepted in J. Am. Ceram. Soc.
	\item Tobias Frueh, I. O. Ozer, S. F. Poterala, Elizabeth R. Kupp, Charles Compson, Joe Atria and Gary L. Messing, "A Critique of Master Sintering Curve Analysis of Sintering Processes," submitted for publication.
\end{itemize}