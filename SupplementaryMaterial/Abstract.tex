\paragraph*{} The vast majority of industrial Al$_{2}$O$_{3}$ applications use powders derived from Bayer feedstocks with purity levels of 99.0 - 99.9\%. This dissertation explores how the extrinsic powder chemistry of Bayer alumina influences densification, microstructure evolution, and fundamental sintering mechanisms.
\paragraph*{} To determine how grain-boundary composition affects the sintering of MgO-free Bayer Al$_{2}$O$_{3}$, dilatometry and sintering kinetics experiments of samples with Na$_{2}$O and SiO$_{2}$ concentrations up to 1029 ppm and 603 ppm, respectively, were performed. It was shown that Na$_{2}$O and SiO$_{2}$ form a nanometer thick siliceous grain boundary film. The estimated viscosities (20 - 400 Pa$\cdot$s) indicate that diffusion greatly depends on the composition of the liquid grain-boundary phase, and densification strongly depends on the Na$_{2}$O/SiO$_{2}$ ratio in the samples. For low Na$_{2}$O/SiO$_{2}$ ratios, densification of Bayer Al$_{2}$O$_{3}$ at 1525$^{\circ}$C is controlled by diffusion of Al$^{3+}$ through the grain-boundary liquid, whereas for high Na$_{2}$O/SiO$_{2}$ ratios, densification can be governed by either the interface reaction of Al$_{2}$O$_{3}$ or diffusion of Al$^{3+}$. Increasing Na$_{2}$O in SiO$_{2}$-doped samples increases diffusion of Al$^{3+}$ and Al$_{2}$O$_{3}$ solubility in the liquid, and thus densification increases by 1-2.5\%. Based on these findings, it is concluded that Bayer Al$_{2}$O$_{3}$ densification is controlled by the Na$_{2}$O to SiO$_{2}$ ratio. 
\paragraph*{} The stages at which MgO influences densification of Al$_{2}$O$_{3}$ were identified by comparing dilatometry measurements and sintering kinetics of MgO-free and 380 ppm MgO-doped Bayer Al$_{2}$O$_{3}$ powder. Using high-resolution TEM it was shown that MgO reduces the grain boundary thickness during final stage sintering. It was shown by EDS that MgO increases the solubility of SiO$_{2}$ in Al$_{2}$O$_{3}$ grains near the boundaries. First-principles DFT calculations demonstrated that the co-dissolution of MgO and SiO$_{2}$ in Al$_{2}$O$_{3}$ is thermodynamically favored over the dissolution of MgO or SiO$_{2}$ individually in Al$_{2}$O$_{3}$. For the first time, it was experimentally demonstrated that removal of SiO$_{2}$ from the grain boundary during final stage sintering is a key process by which MgO enhances the sintering of Al$_{2}$O$_{3}$. 
\paragraph*{} A physical model that describes the dynamic development of grain boundary chemistry and thickness from initial to final stage sintering as a function of chemical and physical powder parameters was developed. The liquid glass phase initially accumulates in the particle contacts due to capillarity and during densification the glass phase distribution changes due to the formation of grain boundaries and grain growth, leading to a dynamic change in grain boundary thickness during densification. Grain boundary thicknesses measured from high-resolution TEM images agree well with predicted film thicknesses. The model shows that the grain boundary thickness is determined by either the concentration of the glass phase in the sample or by an equilibrium film thickness, which is controlled by a balance between attractive and repulsive interparticle forces. Variations in powder chemistry, such as changing the Na$_{2}$O/SiO$_{2}$ ratio or changing the MgO concentration, can affect the grain boundary thickness by influencing the interparticle force balance or by changing the amount of glass phase in the grain boundaries. 
\paragraph*{} The formation of $\beta$-Al$_{2}$O$_{3}$ (Na$_{2}$O$\cdot$11Al$_{2}$O$_{3}$) is sometimes observed during sintering of Bayer Al$_{2}$O$_{3}$, and the conditions and mechanisms of formation were investigated. Formation kinetics showed that the majority of $\beta$-Al$_{2}$O$_{3}$ grains form between 0 and 1 hour at 1525$^{\circ}$C, at densities >90\%. The amount of $\beta$-Al$_{2}$O$_{3}$ increases with increasing Na$_{2}$O and MgO concentrations, and decreases with increasing SiO$_{2}$ concentration. SiO$_{2}$ increases the solubility of Na$_{2}$O in the grain boundaries by forming a glassy film, and thereby counteracts the formation of $\beta$-Al$_{2}$O$_{3}$. At final sintering stage SiO$_{2}$ is removed from the grain boundaries because SiO$_{2}$ and MgO form a solid solution in $\alpha$-Al$_{2}$O$_{3}$, which leads to the supersaturation of grain boundaries and the nucleation and growth of $\beta$-Al$_{2}$O$_{3}$.
\paragraph*{} The Master Sintering Curve (MSC) approach and the factors affecting its accuracy were explored for analyzing the complete sintering profile of Bayer Al$_{2}$O$_{3}$. MSC analysis is sensitive to shrinkage anisotropy, which can be accounted for by directional shrinkage measurements, and to powder chemistry and forming technique effects that result in mechanistic changes that cannot be sufficiently described by MSC analysis. It is concluded that $Q$ should not be interpreted as the sintering activation energy, or used to interpret mechanistic differences between powders since $Q$ includes several mechanisms that influence densification throughout the sintering cycle. Despite these limitations, the MSC approach is a useful and practical tool for predicting the effects of thermal load (i.e. time and temperature) on densification of a specific powder and forming process. 