\chapter{Summary and Future Work}

\section{Summary}
The goal of this work was to gain fundamental understanding on how powder chemistry of commercial Bayer alumina affects densification, microstructural evolution, and sintering mechanisms. A large number of investigations has dealt with the effect of impurities and dopants on the sintering of high purity alumina. However, there is little research about the effects of impurities that are characteristic of Bayer alumina, such as Na$_{2}$O, specifically in the presence of other impurities and dopants that are known to affect sintering of alumina, such as SiO$_{2}$ and MgO. This work systematically investigated cross effects of some impurities and dopants that are characteristic to commercial grade Bayer alumina on densification, microstructural evolution, and fundamental sintering mechanisms. 

\subsection{The Effects of Na$_{2}$O and SiO$_{2}$ on Liquid Phase Sintering of Bayer Al$_{2}$O$_{3}$}
To investigate the effect of Na$_{2}$O and SiO$_{2}$ on sintering of MgO-free Bayer alumina samples with up to 1029 and 603 ppm, respectively, were prepared. Dilatometry and sintering kinetics experiments showed that increasing Na$_{2}$O concentrations led to an increased onset temperature of sintering and results in a lower density up to final stage sintering, but no difference in relative density for sintering times of 3 h or longer at 1525$^{\circ}$C was observed. Increasing SiO$_{2}$ concentrations leads to a significant retardation of densification starting at $\sim$1250$^{\circ}$C and after 8 h at 1525$^{\circ}$C samples with 603 ppm SiO$_{2}$ have 4\% lower density than samples with 103 ppm SiO$_{2}$. In samples with increased SiO$_{2}$ concentrations, such as 603 ppm, the addition of Na$_{2}$O increases the density by 1 - 2.5\% compared to samples with low Na$_{2}$O concentrations. 

The observed sintering behavior was explained by a liquid phase sintering model, where the viscosity of the liquid grain boundary phase, and therefore the diffusion coefficient, are determined by the chemical composition of the liquid grain boundary phase. The viscosity of the grain boundary phase is 2 orders of magnitude lower for samples with high Na$_{2}$O/SiO$_{2}$ ratios, e.g. 0.9 for samples with 529/603 ppm Na$_{2}$O/SiO$_{2}$, than the viscosity of the liquid grain boundary phase of samples with low Na$_{2}$O/SiO$_{2}$ ratios, e.g. 0.1 for samples with 29/603 ppm Na$_{2}$O/SiO$_{2}$. Additionally, the solubility of Al$_{2}$O$_{3}$ in the liquid grain boundary phase is increased for samples with higher Na$_{2}$O concentrations. Both of these factors lead to enhanced densification in samples with higher Na$_{2}$O/SiO$_{2}$ ratios. It is concluded that Bayer Al$_{2}$O$_{3}$ densification can be manipulated by adjusting the Na$_{2}$O to SiO$_{2}$ ratio.

\subsection{Powder Chemistry Effects on the Sintering of MgO-doped Specialty Al$_{2}$O$_{3}$}
MgO is typically added to commercial alumina powder due to its beneficial effect on sintering. To identify the mechanisms responsible for this beneficial effect, 380 ppm MgO-doped Bayer alumina powder with chemical and physical characteristics similar to the MgO-free powder studied in the first part of this dissertation was used and doped to similar chemistries as the MgO-free Bayer alumina. Dilatometry measurements and sintering kinetics of the MgO-free powder and the MgO-doped powder were compared to identify differences in sintering behavior as a result of MgO-doping. In both powders higher SiO$_{2}$ concentrations retard densification significantly, however, this effect is less severe in the MgO-doped powder than in the MgO-free powder. For example the addition of 500 ppm SiO$_{2}$ results in 4\% lower relative density in MgO-free powder samples after sintering at 1525$^{\circ}$C for 8 h, but only 2\% lower relative density in MgO-doped powder samples. While the Na$_{2}$O/SiO$_{2}$ ratio has a similar effect in both powders during initial and intermediate stage sintering, i.e. higher Na$_{2}$O/SiO$_{2}$ ratios increase densification, no effect of the Na$_{2}$O/SiO$_{2}$ ratio was observed for 380 ppm MgO-doped Bayer alumina samples during final stage sintering at densities $\geq$92\%. 

High resolution TEM showed the presence of a liquid phase in the grain boundaries of MgO-free powder samples and in MgO-doped powder samples during intermediate stage sintering (<92\%). During final stage (>92\%) sintering the thickness of the grain boundary phase in the MgO-doped powder samples is significantly reduced compared to MgO-free powder. EDS analysis showed that MgO and SiO$_{2}$ have an increased co-solubility in the alumina lattice where Mg$^{2+}$ and Si$^{4+}$ compensate for each size and charge difference relative to Al$^{3+}$ in the alumina lattice. The thermodynamic stability of such a mechanism was shown by DFT-based first-principles calculations. The reduced amount of SiO$_{2}$ on the grain boundaries of MgO-doped alumina leads to enhanced densification compared to MgO-free alumina because SiO$_{2}$ has been shown to retard densification.

\subsection{Dynamic development of nanometer scale grain boundaries during liquid phase sintering}
It was shown that grain boundary chemistry and structure drastically affect fundamental sintering mechanisms and therefore it is crucial to understand how grain boundaries change during densification. A physical model was developed to describe the dynamic development of grain boundaries in Bayer alumina during densification. The liquid glass phase concentration that forms during sintering is a function of powder chemistry and the glass phase concentration changes during densification because the temperature of the sample changes during heating. The liquid glass phase initially accumulates in the particle contacts due to capillarity, and then the distribution of the glass phase changes during sintering due to the formation of grain boundaries and grain growth, which leads to the observed dynamic change in grain boundary thickness. The model predicts that the grain boundary thickness can be governed by either the glass phase concentration or by an equilibrium grain boundary thickness. If sufficient glass phase is present to form an intergranular film of equilibrium thickness, then the balance between attractive and repulsive colloidal forces determines the grain boundary thickness, and for grain boundary films >1 nm the main contributions to this force balance are attractive capillary forces and repulsive structural disjoining forces. 

Changes in Na$_{2}$O/SiO$_{2}$ ratio and/or the addition of MgO changes the concentration of liquid phase in the sample and the interparticle force balance, and, therefore, changes the grain boundary thickness. It was shown that grain boundary thicknesses measured by high resolution TEM agree well with predicted film thicknesses as a function of powder chemistry.  

\subsection{Second phase formation in Bayer alumina}
Bayer alumina has higher impurity concentrations than ultra-high pure aluminas, which makes these powders more prone to second phase formation, and thus the conditions of formation and the formation process of second phases in Bayer alumina were explored as a function of powder chemistry. XRD showed that $\beta$-Al$_{2}$O$_{3}$ was the only second phase that forms for the range of chemistries explored in this dissertation. No $\beta$-Al$_{2}$O$_{3}$ was observed in samples with low Na$_{2}$O and MgO concentrations such as 29 and 2 ppm, respectively. Only a small number of $\beta$-Al$_{2}$O$_{3}$grains form when the Na$_{2}$O concentration is $\leq$529 ppm. However, there was a significant increase in the amount of $\beta$-Al$_{2}$O$_{3}$ observed when the Na$_{2}$O concentration was increased to 1029 ppm. Increasing the MgO concentration in samples with 29 ppm Na$_{2}$O increases the number of $\beta$-Al$_{2}$O$_{3}$ grains up to 502 ppm, but no further increase in number density of $\beta$-Al$_{2}$O$_{3}$ grains is observed when the MgO concentration is increased to 1002 ppm. In general, increasing Na$_{2}$O and MgO concentrations leads to an increase in number density and a decrease in $\beta$-Al$_{2}$O$_{3}$ grain size, but increasing the SiO$_{2}$ concentration results in a decrease in number density and increase in $\beta$-Al$_{2}$O$_{3}$ grain size. 

It was shown that SiO$_{2}$ and Na$_{2}$O form a liquid grain boundary phase during sintering of alumina, and the amount of $\beta$-Al$_{2}$O$_{3}$ that forms is determined by the Na$_{2}$O-supersaturation of the liquid grain boundary phase, i.e. the Na$_{2}$O/SiO$_{2}$ ratio in the sample. Formation kinetics showed that the majority of beta alumina form within the first hour at 1525$^{\circ}$C, which is within the same time as when the grain boundary thickness decreases due to the proposed MgO- and SiO$_{2}$ co-dissolution mechanism into the alumina grain. It was concluded that MgO increases the amount of $\beta$-Al$_{2}$O$_{3}$ because the co-dissolution of MgO and SiO$_{2}$ into the alumina grain lowers the SiO$_{2}$ concentration in the glassy grain boundaries during final stage sintering. As a result, the Na$_{2}$O/SiO$_{2}$ ratio in the grain boundaries increases, which leads to a higher Na$_{2}$O-supersaturation and to an enhanced formation of $\beta$-Al$_{2}$O$_{3}$. 

\subsection{A Critique of Master Sintering Curve Analysis of Sintering Processes}
The objective of this work was to explore if the Master Sintering Curve (MSC) approach can be used as a tool to identify powder chemistry effects on the sintering of Bayer alumina in a fundamental, predictive way. MSC analysis results in an apparent activation energy $Q$ and the MSC itself. It was shown that processing history and powder chemistry drastically affect the value of $Q$ and the shape of the MSC but no direct correlation was found between powder chemistry and the MSC shape or $Q$ since changes in powder chemistry can cause complex variations in sintering mechanisms. Likewise, $n$ direct correlation was found between forming technique, the resulting green density, and $Q$ or the shape of the MSC because factors such as pore size and pore size distribution, which are known to influence densification, are not accounted for in MSC analysis. Finally, the applicability of MSC at high densities (>90\%) is limited by changes in microstructural evolution that occur as a function of densification rate, but are not accounted for in MSC analysis.

It was shown that MSC analysis can be a useful tool for predicting the sintering behavior of a specific powder after its $Q$ and MSC are accurately determined. For example, shrinkage anisotropy has to be accounted for to determine accurate $Q$ values and MSCs. Under these conditions MSC analysis is only a useful and practical tool to predict sintering behavior, but again fails to analyze fundamental sintering processes or changes that result from differences in chemical or physical variations of the sintering material.

\section{Future work}
This work focused on the effects and cross effects of Na$_{2}$O, SiO$_{2}$, and MgO on the sintering of commercial Bayer alumina and is therefore a step towards a fundamental understanding of the interplay of multiple dopants and impurities during sintering of commercial Bayer alumina. Several paths for future research are proposed below.

The effects of additional impurities that can be found in commercial powders, specifically CaO, should be investigated. Multiple studies have investigated the effect of CaO on the sintering of ultra-high purity alumina, however, it is crucial to consider the presence of other impurities and dopants. The importance of considering the presence of other impurities and dopants was shown in this work; for example, the effect of Na$_{2}$O on the sintering of Bayer alumina strongly depends on the SiO$_{2}$ concentration and the Na$_{2}$O/SiO$_{2}$ ratio, and the effect of MgO depends on the SiO$_{2}$ concentration. Similar relations could exist between CaO and other impurities and dopants, and it is of great interest to identify such effects. Based on the present investigation and literature reports a variety of hypotheses of the effect of CaO on the sintering of Bayer alumina are reasonable. Initial and intermediate stage sintering in Bayer alumina are governed by a liquid phase sintering mechanism, due to the presence of SiO$_{2}$, which forms a siliceous glass phase in the grain boundaries. At these sintering stages Na$_{2}$O and MgO were shown to affect the properties of the liquid grain boundary phase by acting as a network modifier. It is likely that CaO behaves in a similar manner. However, calcium silicate glass melts are known to exhibit liquid-liquid miscibility gaps that are influenced by the MgO, Na$_{2}$O, and Al$_{2}$O$_{3}$ concentrations \cite{Kingery1983}. This suggests that glass phases of different compositions can be present at the same time, which could significantly affect sintering, since diffusion is controlled by viscosity and, therefore, the composition of the liquid phase. Furthermore, different concentrations in CaO can affect second phase formation. For example, CaO by itself forms calcium hexaluminate (CaO*6Al$_{2}$O$_{3}$) in alumina, which can form a solid solution with $\beta$-Al$_{2}$O$_{3}$ in the high alumina region of the Na$_{2}$O-SiO$_{2}$-CaO-Al$_{2}$O$_{3}$ system \cite{Vries1969}. Different ratios and concentrations of impurities and dopants might lead to different amounts and types of second phases, and, therefore, the influence of CaO on the formation of second phases should be investigated.

Grain boundaries of ceramics have been extensively researched over the past decades because their structure and chemistry determine sintering mechanisms and many properties of ceramics, such as corrosion, mechanical, and electrical properties. In this work the chemical composition and structure of grain boundaries was investigated to gain insight into sintering mechanisms and to explain the observed sintering behavior. Therefore, samples were quenched from the sintering temperature to room temperature to "freeze" the chemistry and structure at the sintering temperature. However, ceramic parts for most applications are cooled at much slower rates, which can drastically affect grain boundary structure, thickness, and chemistry, and the grain boundaries investigated in this work may not be representative for grain boundaries in slow cooled Bayer alumina ceramic parts. The influence of cooling rate on the chemistry and structure of grain boundaries of Bayer alumina should be investigated to gain insight into how properties are affected by different powder chemistries and grain boundary characteristics. The development of time-temperature-transformation (TTT) diagrams for grain boundaries \cite{Cantwell2016} as a function of powder chemistry could be a powerful tool to design grain boundaries and tailor the properties of alumina ceramics. 

In this dissertation a physical model was developed to predict the dynamic development of grain boundaries during densification. A potential application for such predictions is to determine the applicability of different sintering models, i.e. solid-state sintering or liquid phase sintering. For samples with high SiO$_{2}$ concentrations, such as 603 ppm, it was demonstrated that sintering at 1525$^{\circ}$C can be analyzed using a liquid phase sintering model, and for samples with low SiO$_{2}$ concentrations, such as 103 ppm, the grain boundary structure and chemistry at 1525$^{\circ}$C suggests that a solid state sintering model is more appropriate to analyze sintering. However, it was also demonstrated that the structure and chemistry of the grain boundaries are a function of additional parameters, such as temperature, relative density, powder chemistry, and grain size. For example, the grain boundary thickness first decreases and then increases as a function of relative density, which implies that for certain glass phase concentrations there can be a transition from liquid phase sintering to solid state sintering and vice versa. Another example is the effect of MgO on final stage sintering of Bayer alumina. It was shown that during initial and intermediate stages alumina with 582 ppm SiO$_{2}$ densifies by a liquid phase sintering mechanism. However, if MgO is present then MgO and SiO$_{2}$ form a solid solution in the alumina grains during final stage sintering, which reduces the amount of SiO$_{2}$ in the grain boundaries, and, depending on the MgO concentration, sintering should be analyzed using a solid state sintering model. This described mechanism suggests that transitions from liquid phase sintering to solid state sintering and vice versa are possible during densification. It would be interesting to investigate in greater detail the effects of such transitions on the sintering process. 
