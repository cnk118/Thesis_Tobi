\chapter{Summary and Future Work}

\section{Summary}
The goal of this work was to gain fundamental understanding on how powder chemistry of commercial Bayer alumina affects the sintering behavior and mechanisms. A large number of investigations have dealt with the effect of impurities and dopants on the sintering of alumina. However, only little research has been done on impurities that are characteristic to Bayer alumina, such as Na$_{2}$O, specifically in the presence of other impurities and dopants that are known to affect sintering of alumina, such as SiO$_{2}$ and MgO. This work systematically investigated cross effects of some impurities and dopants that are characteristic to commercial grade Bayer alumina on sintering behavior and fundamental sintering mechanisms. 

\subsection{The Effects of Na$_{2}$O and SiO$_{2}$ on Liquid Phase Sintering of Bayer Al$_{2}$O$_{3}$}
In the first part of this dissertation the effect of Na$_{2}$O and SiO$_{2}$ on sintering of MgO-free Bayer alumina was explored at levels of up to 1029 and 529 ppm, respectively. Dilatometry and sintering kinetics experiments showed that increasing Na$_{2}$O concentrations lead to an increased onset temperature of sintering and leads to lower densities up to final stage sintering, but no difference in relative density for sintering times of 3 h or longer at 1525$^{\circ}$C was observed. Increasing SiO$_{2}$ concentrations leads to a significant retardation of densification starting at $\sim$1250°C and after 8 h at 1525$^{\circ}$C samples with 603 ppm SiO$_{2}$ have 4\% lower density than samples with 103 ppm SiO$_{2}$. In samples with increased SiO$_{2}$ concentrations, such as 603 ppm, the addition of Na$_{2}$O increases the density by 1 - 2.5\% compared to samples with low Na$_{2}$O concentrations. 

The observed sintering behavior was explained by a liquid phase sintering model, where the viscosity of the liquid grain boundary phase, and therefore the diffusion coefficient, are determined by the chemical composition of the liquid grain boundary phase. For high Na$_{2}$O/SiO$_{2}$ ratios, e.g. 0.9 for samples with 529/603 ppm Na$_{2}$O/SiO$_{2}$, the viscosity of the grain boundary phase is 2 orders of magnitude lower than the viscosity of the liquid grain boundary phase of samples with low Na$_{2}$O/SiO$_{2}$ ratios, e.g. 0.1 for samples with 29/603 ppm Na$_{2}$O/SiO$_{2}$. Furthermore, the solubility of Al$_{2}$O$_{3}$ in the liquid grain boundary phase is increased for samples with higher Na$_{2}$O concentrations. Both of these factors lead to enhanced densification in samples with higher Na$_{2}$O/SiO$_{2}$ ratios. 

\subsection{Powder Chemistry Effects on the Sintering of MgO-doped Specialty Al$_{2}$O$_{3}$}
In a next step an MgO-doped (380 ppm) Bayer alumina powder with chemical and physical characteristics similar to the MgO-free powder in the first part was used and doped to similar chemistries as the MgO-free Bayer alumina. Dilatometry measurements and sintering kinetics of the MgO-free powder and the MgO-doped powder were compared to identify differences in sintering behavior as a result of MgO-doping. In both powders higher SiO$_{2}$ concentrations retard densification significantly, however, this effect is less severe in the MgO-doped powder than in the MgO-free powder. E.g. the addition of 500 ppm SiO$_{2}$ results in 4\% lower relative density in MgO-free powder samples after sintering at 1525$^{\circ}$C for 8 h, but only 2\% lower relative density in MgO-doped powder samples. While the Na$_{2}$O/SiO$_{2}$ ratio has a similar effect in both powders during initial and intermediate stage sintering, i.e. higher Na$_{2}$O/SiO$_{2}$ ratios increase densification, no effect of the Na$_{2}$O/SiO$_{2}$ ratio was observed during final stage sintering at densities $\geq$92\%. 

High resolution TEM showed the presence of a liquid phase in the grain boundaries of MgO-free powder samples and in MgO-doped powder samples during intermediate stage sintering. During final stage sintering the thickness of the grain boundary phase in the MgO-doped powder samples is significantly reduced compared to MgO-free powder. EDS analysis suggested that MgO and SiO$_{2}$ have an increased co-solubility in the alumina lattice where Mg$^{2+}$ and Si$^{4+}$ compensate for each others charge and strain. The thermodynamic stability of such a mechanism was shown by DFT-based first-principles calculations. This co-dissolution process reduces the amount of SiO$_{2}$ on the grain boundaries of MgO-doped Bayer alumina and, therefore, leads to enhanced densification. 

\subsection{Dynamic development of nanometer scale grain boundaries during liquid phase sintering}
In the third part of this work a physical model was developed to describe the dynamic development of grain boundaries in Bayer alumina during densification. The liquid glass phase concentration that forms during sintering is a function of powder chemistry and changes during heating. The liquid glass phase initially accumulates in the particle contacts due to capillarity, and then the distribution of the glass phase changes during sintering due to the formation of grain boundaries and grain growth, which leads to the observed dynamic change in grain boundary thickness. The model predicts that the grain boundary thickness can be governed by either the glass phase concentration or by an equilibrium grain boundary thickness. If sufficient glass phase is present to form an intergranular film of equilibrium thickness, a balance between attractive and repulsive colloidal forces determines the grain boundary thickness, and it was demonstrated that for grain boundary films >1 nm the main contributions to this force balance are attractive capillary forces and repulsive structural disjoining forces. If the glass phase concentration is insufficient to form an intergranular film of equilibrium thickness the grain boundary thickness is determined by the glass phase concentration. 

Changes in powder chemistry, such as changing the Na$_{2}$O/SiO$_{2}$ ratio or the addition of MgO changes both, the concentration of liquid phase in the sample and the interparticle force balance, and, therefore, changes the grain boundary thickness. Predicted grain boundary thicknesses for different powder chemistries agree well with grain boundary thicknesses measured by high resolution TEM.  

\subsection{Second phase formation in Bayer alumina}
The formation of a second phase due to the presence of impurities and dopants in Bayer alumina was observed during sintering, which was investigated in Chapter 4. XRD showed that the second phase that forms in samples with chemistries investigated in this work was beta alumina. No beta alumina was observed in samples with low Na$_{2}$O and MgO concentrations such as 29 and 2 ppm, respectively. Only a small number of beta alumina grains form when the Na$_{2}$O concentration is increased to 529 ppm, but a significant increase in the amount of beta alumina was observed when the Na$_{2}$O concentration was increased to 1029 ppm. Increasing the MgO concentration in samples with 29 ppm Na$_{2}$O increases the number of beta alumina grains up until 502 ppm, but no further increase in beta alumina grains is observed when the MgO concentration is increased to 1002 ppm. In general, an increase in number and decrease in size of beta alumina grain was observed with increasing Na$_{2}$O and MgO concentrations, but a decrease in number and increase in size was observed with increasing SiO$_{2}$ concentration. 

The nucleation frequency of beta alumina grains increases with increasing Na$_{2}$O concentration, decreasing SiO$_{2}$ concentrations, and increasing Na$_{2}$O/SiO$_{2}$ ratio due to the higher supersaturation of the grain boundaries. Formation kinetics showed that the majority of beta alumina forms within the first hour at 1525$^{\circ}$C, which is the same time as when the grain boundary thickness decreases due to the proposed MgO$^{-}$ and SiO$_{2}$ co-dissolution mechanism described in Chapter 3. MgO increases the amount of beta alumina that forms due to the MgO + SiO$_{2}$ co-dissolution mechanism described in Chapter 3. The removal of SiO$_{2}$ from the grain boundaries increases the Na$_{2}$O/SiO$_{2}$ ratio in the grain boundaries during final stage sintering and, therefore, to increases the Na$_{2}$O-supersaturation leading to an enhanced nucleation of beta alumina. 

\subsection{A Critique of Master Sintering Curve Analysis of Sintering Processes}
Goal of this work was to identify powder chemistry effects on the sintering of Bayer alumina and in the final part of this dissertation it was explored if the Master Sintering Curve (MSC) approach can be used as a tool to quantify such changes in a meaningful way. MSC analysis results in an apparent activation energy $Q$ and the MSC itself. The shape of MSC is determined by the microstructural evolution and can be quantified by a parameter $C$, which includes all microstructural parameters. $Q$ is a fitting parameter that describes the activation energy of a variety of processes that occur during densification. $Q$ does not correspond to a specific sintering mechanism and, therefore, has no physical meaning. 

Processing history and powder chemistry drastically affect $Q$ and the shape of the MSC but no direct correlation was found between powder chemistry and the MSC shape or $Q$ since changes in powder chemistry can cause complex variations in sintering mechanisms. No direct correlation was found between forming technique, the resulting green density, and $Q$ or the shape of the MSC because factors such as pore size and pore size distribution, which are known to influence densification, are not accounted for in MSC analysis. 

It was shown that MSC analysis is useful for predicting the sintering behavior of a specific powder after its $Q$ and MSC are determined. To obtain accurate $Q$ values and MSCs shrinkage anisotropy has to be taken into account. The applicability of MSC at high densities (>90\%) is limited by changes in microstructural evolution that occur as a function of densification rate, but are not accounted for in MSC analysis. MSC analysis is a useful and practical tool to predict sintering behavior, but fails to analyze fundamental sintering processes or changes that result from differences in chemical or physical variations of the sinter.

\section{Future work}
This work focused on the effects and cross effects of Na$_{2}$O, SiO$_{2}$, and MgO on the sintering of commercial Bayer alumina and is therefore a step towards a fundamental understanding of the interplay of multiple dopants and impurities during sintering of commercial Bayer alumina. To expand on this work there are several paths for future work.

The effects of additional impurities that can be found in commercial powders, specifically CaO, should be investigated. Multiple studies have investigated the effect of CaO on the sintering of alumina, however, it is crucial to consider the presence of other impurities and dopants. The importance of considering the presence of other impurities and dopants was shown in this work; for example, the effect of Na$_{2}$O on the sintering of Bayer alumina strongly depends on the SiO$_{2}$ concentration and the Na$_{2}$O/SiO$_{2}$ ratio, and the effect of MgO depends on the SiO$_{2}$ concentration. Similar relations could exist between CaO and other impurities and dopants, and it is of great interest to identify such effects. Based on the present investigation and literature reports a variety of hypotheses of the effect of CaO on the sintering of Bayer alumina are reasonable. Initial and intermediate stage sintering in Bayer alumina are governed by a liquid phase sintering mechanism, due to the presence of SiO$_{2}$, which forms a siliceous glass phase in the grain boundaries. At these sintering stages Na$_{2}$O and MgO were shown to affect the properties of the liquid grain boundary phase by acting as a network modifier. It is likely that CaO behaves in a similar manner. However, calcium silicate glass melts are known to exhibit liquid-liquid miscibility gaps that are influenced by the MgO, Na$_{2}$O, and Al$_{2}$O$_{3}$ concentrations \cite{Kingery1983}. This means that glass phases of different compositions can be present at the same time, which could significantly affect sintering, since diffusion is controlled by viscosity and, therefore, the composition of the liquid phase. Furthermore, different concentrations in CaO can affect second phase formation. For example, CaO by itself forms calcium hexaluminate (CaO*6Al$_{2}$O$_{3}$) in alumina, which can form a solid solution with $\beta$-Al$_{2}$O$_{3}$ in the high alumina region of the Na$_{2}$O-SiO$_{2}$-CaO-Al$_{2}$O$_{3}$ system \cite{Vries1969}. Different ratios and concentrations of impurities and dopants might lead to different amounts and types of second phases, and, therefore, the influence of CaO on the formation of second phases should be investigated.

In this work the chemical composition and structure of grain boundaries was investigated to gain insight into sintering mechanisms and to explain the observed sintering behavior. Since the heating and cooling schedule significantly influence the structure and chemistry of grain boundaries, samples in this work were quenched from the sintering temperature to room temperature to "freeze" the chemistry and structure during sintering. Additionally to their effect on sintering grain boundaries significantly influence properties of ceramics, such as corrosion, mechanical, and electrical properties. However, ceramic parts for applications are typically cooled at much slower rates, which can affect grain boundary structure, thickness, and chemistry. Therefore, the influence of cooling rate on the chemistry and structure of grain boundaries of Bayer alumina should be investigated. The development of time-temperature-transformation (TTT) diagrams for grain boundaries \cite{Cantwell2016} as a function of powder chemistry could be a powerful tool to design grain boundaries and tailor the properties of alumina ceramics. 

In Chapter 4 of this dissertation a physical model was developed to predict the dynamic development of grain boundaries during densification. A potential application for such predictions is to determine the applicability of different sintering models, i.e. solid-state sintering or liquid phase sintering. For samples with high SiO$_{2}$ concentrations, such as 603 ppm, it was demonstrated that sintering at 1525$^{\circ}$C can be analyzed using a liquid phase sintering model, and for samples with low SiO$_{2}$ concentrations, such as 103 ppm, the grain boundary structure and chemistry at 1525$^{\circ}$C suggests that a solid state sintering model is more appropriate to analyze sintering. However, it was also demonstrated in Chapter 4 that the structure and chemistry of the grain boundaries are a function of additional parameters, such as temperature, relative density, powder chemistry, and grain size. For example, the grain boundary thickness first decreases and then increases as a function of relative density, which implies that for certain glass phase concentrations there can be a transition from liquid phase sintering to solid state sintering and vice versa. Another example is the effect of MgO on final stage sintering of Bayer alumina. It was shown that during initial and intermediate stage sintering samples with 582 ppm SiO$_{2}$ densify by a liquid phase sintering mechanism, but if MgO is present in sample MgO and SiO$_{2}$ form a solid solution in the alumina grains during final stage sintering, which reduces the amount of SiO$_{2}$ in the grain boundaries, and, depending on the MgO concentration, sintering should be analyzed using a solid state sintering model. This described mechanism suggest that during densification transitions from liquid phase sintering to solid state sintering are possible. It would be interesting to investigate such transitions from solid state sintering to liquid phase sintering and vice versa and their effects on sintering in greater detail. 
